%%% ====================================================================
%%% Início da parte textual do documento.


%%% Configuração do espaçamento entre títulos e texto
\setlength{\afterchapskip}{1.5cm minus \baselineskip}


\chapter{Introdução}
\label{cha:motivacao}
Aplicações educacionais são \textit{softwares} que promovem o aperfeiçoamento do processo acadêmico de aprendizagem. Tais sistemas proporcionam a seus utilizadores uma perspectiva mais ampla e tangível de informações, trazendo em conjunto funcionalidades suplementares que tornam a relação de aprendizado mais analítica e eficiente. \citeonline{karolcik2015comprehensive} afirmam que o \textit{Software} Educacional deve ser simples e intuitivo, ao mesmo tempo oferecendo ao usuário um alto nível de comodidade.

Na metodologia de aprendizado, a relação entre escola e família compreende um elemento de vital importância, impactando diretamente na vida acadêmica do aluno. Sua dependência é tal que \citeonline{dessen2007familia} a definem como fundamental no processo de desenvolvimento do indivíduo, impulsionando diretamente seu crescimento intelectual, emocional, físico e social.

O progresso da tecnologia tem condicionado recursos cada vez mais ágeis e interativos, capazes apresentar dados complexos como informação inteligível e comunicativa. Para \citeonline{silva2014novas}, o meio educacional tem reconhecido a importância da tecnologia em seu contexto, e o quão rico é este instrumento para o ensino. \citeonline{silva2016uso} afirmam que estamos diante de um cenário em que a tecnologia digital permite novas formas de ensino, devido a sua riqueza de modelos e conteúdos.

Através disso, podemos destacar a relevância das tecnologias móveis como um elemento de suma importância neste contexto, viabilizando acesso ilimitado e instantâneo à informação. \citeonline{qi2012research} elencam a tamanha aceitação desta tecnologia e o quanto ela se faz presente na vida de milhares de pessoas, substituindo os computadores pessoais como o principal meio de acesso à internet e como forma de trabalho.

O trabalho em questão se insere neste meio, propondo o desenvolvimento de um aplicativo \textit{mobile} que atue como facilitador na relação entre alunos, pais e escola, objetivando um maior comprometimento das partes no processo de aprendizagem. Conjuntamente, é proposto a elaboração de um módulo baseado em redes neurais que, com base na análise dos dados dos alunos, apresentará uma sugestão quanto à aptidão de um aluno, levando em conta as áreas de conhecimento definidas pelo Novo Ensino Médio.

Dentre as principais funcionalidades do aplicativo, compreendem:
\begin{itemize}
	\item a disponibilização de dados acadêmicos em tempo real, tais como grade horária de aulas, registro de atividades, conteúdos ministrados, frequência e notas;
	\item a visualização do calendário acadêmico;
	\item a visualização de mensagens da escola e dos professores;
\end{itemize}

\subsection{OBJETIVOS}

\subsubsection{Geral}

Elaborar um aplicativo \textit{mobile} para a promoção de integração de pais, alunos e escola no contexto educacional, permitindo a visualização de dados acadêmicos em tempo real;

\subsubsection{Específicos}

\begin{enumerate}
	\item Elaborar e implantar um \textit{web service} que irá buscar informações dos alunos em uma base de dados;
	\item Elaborar um aplicativo mobile que utilize os dados fornecidos através do \textit{web service};
	\item Elaborar um módulo de redes neurais para classificação de alunos com base nas áreas de conhecimento do Novo Ensino Médio;
\end{enumerate}
