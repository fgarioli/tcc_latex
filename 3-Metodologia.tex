\chapter{\textbf{Metodologia de Pesquisa}} % Este comando é utilizado para criar capítulos
\sloppy % Corrige estouro de linhas

\section{Aplicativo}

O escopo da aplicação foi definido para contemplar uma base de dados já existente de um sistema de gestão educacional já implantado em um município, administrado por uma empresa parceira, a qual cedeu o acesso à mesma para aproveitamento na construção do \textit{software}. Após este processo, o mesmo ficará disponível para utilização por parte da população do município, bem como toda a sua codificação.

Para a construção do mesmo, foi utilizado como referência de metodologia de desenvolvimento o modelo aplicado na disciplina de Laboratório de Engenharia de Software do curso de Sistemas de Informação do Ifes Campus Cachoeiro, baseado no \textit{Rational Unified Process} (RUP), representado na figura \ref{figura:rup}. 

\begin{figure}[H]
	\caption{Processo de desenvolvimento de \textit{software} baseado no RUP.}
	\centering % para centralizarmos a figura
	\includegraphics[width=16cm]{resources/pds_rup.png} % leia abaixo
	\label{figura:rup}
	\captionsetup{singlelinecheck = false, format= hang, justification=raggedright, labelsep=space, width=16cm}
	\caption*{\footnotesize Fonte: \citeonline{rupLes}.}
\end{figure}

Para tal, os seguintes passos descritos no modelo (figura \ref{figura:rup}) foram executados para a elaboração do \textit{sofware}:

\begin{enumerate}
   \item Levantar Requisitos (Visão Geral): identificar descrever de forma geral o funcionamento do aplicativo e suas principais funções;
   \item Elaborar modelo conceitual: identificar classes e relacionamentos concernentes ao domínio (nesta fase não se faz necessário a inclusão dos atributos nas classes);
   \item Levantar Requisitos: elaborar documento de requisitos contendo o detalhamento de todas as funcionalidades do aplicativo;
   \item Organizar Requisitos: elaborar documento de organização de requisitos separando todos os requisitos em processos de negócios e relatórios/listagens;
%   \item Elaborar banco de dados: 
%   \item Elaborar interface gráfica: estruturar todas as telas do aplicativo;
   \item Implementar: codificar as classes necessárias para implementação das funcionalidades de processos de negócio e listagens;
   \item Testar: realizar testes de todas as funcionalidades implementadas;
   \item Implantar: instalar o \textit{web service} no servidor de produção, gerar o arquivo instalável do aplicativo e publicar nas respectivas lojas de aplicativos;
\end{enumerate}

\section{Módulo de classificação baseado em Redes Neurais}

Considerando as áreas de conhecimento definidas pelo Novo Ensino Médio \cite{lei13415}, foi realizado a construção de um módulo que classifica, levando em consideração dados de um aluno, as áreas mais próximas ao seu perfil. Para tanto, foram realizadas as seguintes etapas no processo de desenvolvimento:

\begin{enumerate}
   \item Selecionar variáveis para o treinamento da rede;
   \item Preparar o arquivo de treinamento com base nas variáveis escolhidas;
   \item Realizar o treinamento da rede;
   \item Integrar o módulo junto ao \textit{web service} e ao aplicativo;
\end{enumerate}

Para a construção do módulo de classificação do aplicativo foi utilizado o software \textit{Weka} para realizar o treinamento da rede e extrair os resultados do mesmo. Para tanto, a biblioteca utiliza um arquivo do tipo ARFF para descrever os os atributos, as possibilidades de classificação, e os respectivos exemplos de classificação utilizados no treinamento em questão. Sua estrutura básica é representada na figura \ref{figura:arff}.

\begin{figure}[H]
	\caption{Sintaxe básica de um arquivo de treinamento utilizado pelo Weka.}
	\centering % para centralizarmos a figura
	\includegraphics[width=12cm]{resources/arff.png} % leia abaixo
	\label{figura:arff}
	\captionsetup{singlelinecheck = false, format= hang, justification=raggedright, labelsep=space, width=12cm}
	\caption*{\footnotesize Fonte: O Autor.}
\end{figure}

\section{Arquitetura da Aplicação}

Toda a aplicação foi construída baseado na arquitetura REST \cite{fielding2000}, seguindo a estrutura representada na figura \ref{figura:arqu_basica}. Neste modelo é construído uma API REST (web serivice), que busca as informações referentes aos alunos na base de dados, trata os mesmos, e as retorna no formato JSON para um aplicativo instalado em \textit{smartphones}, que poderão utilizar os sistema Android ou IOS.

\begin{figure}[H]
	\caption{Arquitetura básica da Aplicação.}
	\centering % para centralizarmos a figura
	\includegraphics[width=12cm]{resources/esquema_web_service.png} % leia abaixo
	\label{figura:arqu_basica}
	\captionsetup{singlelinecheck = false, format= hang, justification=raggedright, labelsep=space, width=12cm}
	\caption*{\footnotesize Fonte: O Autor.}
\end{figure}

\section{Ambiente de desenvolvimento}

As seguintes ferramentas e \textit{softwares} foram utilizados no processo de desenvolvimento:

\begin{itemize}
    \item \textit{Java SE Development Kit} (versão 8 \textit{update} 221);
    \item \textit{Ionic Framework} (versão 4);
    \item Servidor \textit{Web Payara} (versão 5.191);
    \item \textit{Netbeans IDE} (versão 11.1);
    \item \textit{Visual Studio Code} (versão 1.39.1);
    \item \textit{Postman} (versão 7.2);
\end{itemize}