\chapter{\textbf{Conclusões gerais e trabalhos futuros}} % Este comando é utilizado para criar capítulos

O presente trabalho proporcionou a construção de um aplicativo \textit{mobile} que visa estimular uma maior participação de responsáveis na vida acadêmica de alunos de uma rede municipal de ensino, realizando a integração dos mesmos com a escola. Para tal, o \textit{software} permitiu a visualização em tempo real de dados acadêmicos dos alunos pelos usuários, bem como de mensagens provenientes das escolas e professores. Conjuntamente, foi disponibilizado a possibilidade de classificação de um aluno levando em consideração as áreas de conhecimento do novo ensino médio, visando o auxílio no conhecimento das aptidões acadêmicas do mesmo.

O aplicativo foi avaliado por alunos do ensino médio do Ifes Campus Cachoeiro, obtendo uma média de 88\% de aprovação nos itens considerados através do formulário de pesquisa aplicado. A rede neural elaborada obteve um percentual de 94\% de acerto na classificação de alunos, considerando dados simulados.

Como trabalhos futuros, verificou-se a realização do treinamento da rede neural, considerando dados de classificação real dos alunos, após a implementação do novo ensino médio. Ademais, verificou-se também a possibilidade de implementação, junto à funcionalidade de mensagens, da comunicação por parte dos responsáveis com a escola e com os professores do aluno.

Levando em consideração as tecnologias utilizadas no processo de desenvolvimento, a plataforma construída permite facilmente que possíveis novas funcionalidades sejam incorporadas a ela de maneira ágil.