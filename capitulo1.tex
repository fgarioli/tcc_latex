%%% ====================================================================
%%% Início da parte textual do documento.


%%% Configuração do espaçamento entre títulos e texto
\setlength{\afterchapskip}{1.5cm minus \baselineskip}


\chapter{Introdução}
\label{cha:motivacao}
Aplicações educacionais são softwares que promovem o aperfeiçoamento do processo acadêmico de aprendizagem. Tais sistemas proporcionam a seus utilizadores uma perspectiva mais ampla e tangível de informações, trazendo em conjunto funcionalidades suplementares que tornam a relação de aprendizado mais analítica e eficiente. \citeonline{karolcik2015comprehensive} afirmam que o Software Educacional deve ser simples e intuitivo, ao mesmo tempo oferecendo ao usuário um alto nível de comodidade.

Na metodologia de aprendizado, a relação entre escola e família compreende um elemento de vital importância, impactando diretamente na vida acadêmica do aluno. Sua dependência é tal que \citeonline{dessen2007familia} a definem como fundamental no processo de desenvolvimento do indivíduo, impulsionando diretamente seu crescimento intelectual, emocional, físico e social.

O progresso da tecnologia tem condicionado recursos cada vez mais ágeis e interativos, capazes apresentar dados complexos como informação inteligível e comunicativa. Para \citeonline{silva2014novas}, o meio educacional tem reconhecido a importância da tecnologia em seu contexto, e o quão rico é este instrumento para o ensino. \citeonline{silva2016uso} afirmam que estamos diante de um cenário em que a tecnologia digital permite novas formas de ensino, devido a sua riqueza de modelos e conteúdos.

Através disso, podemos destacar a relevância das tecnologias móveis como um elemento de suma importância neste contexto, viabilizando acesso ilimitado e instantâneo à informação. \citeonline{qi2012research} elencam a tamanha aceitação desta tecnologia e o quanto ela se faz presente na vida de milhares de pessoas, substituindo os computadores pessoais como o principal meio de acesso à internet e como forma de trabalho.

O trabalho em questão se insere neste meio, propondo o desenvolvimento de um aplicativo móvel que atue como facilitador na relação entre alunos, pais e escola, objetivando um maior comprometimento das partes no processo de aprendizagem. Dentre suas principais funções podemos destacar:
\begin{itemize}
	\item Disponibilização de dados acadêmicos em tempo real;
	\item Visualização de mensagens da escola e dos professores;
	\item Visualização de tarefas e conteúdos de aulas;
	\item Acompanhamento da frequência e das notas do aluno;
	\item Fornececimento de um ambiente interativo auxiliar para a gestão da vida acadêmica;
\end{itemize}

Neste sentido, propõe-se conjuntamente a utilização de dados dos alunos de para uma análise de padrões, forma a gerar informações auxiliares para alunos e pais. Tal análise se utilizará dos recursos de redes neurais para apontar possíveis fatores capazes de contribuir para o aprendizado do aluno. Deste modo, serão selecionados, com o auxílio de um pedagogo através de coorientação,  parâmetros iniciais básicos, que assistirão ao treinamento da rede proposta, que ocorrerá em seguida.